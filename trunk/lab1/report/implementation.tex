The implementation for Lab 1 is done using MATLAB's class structures to
maximize the reusability and allow for experiementation beyond the requirements
of the lab. This is discussed further in Chapter \ref{cha:results}.

\section{Properties and class functions}
MATLAB classes were created for parametric and non-parametric classifiers. The
classes, {\tt ParametricClass} and {\tt NonParametricClass} are presented in
Appendix \ref{cha:code}.
\subsection{Properties}
The two classes store properties related to the pattern recognition problems
they represent. The {\tt ParametricClass} stores for a class $A$ the values of
$\mu_A$, $\Sigma_A$ and $p(A)$. The {\tt NonParametricClass} stores a cluster
of $n$ points in a Gaussian distribution with the parameters $\mu$
and $\Sigma$.

\subsection{Class functions}
Each class provides methods for calculating the various distance measures
associated with the type of problem that it represents. The {\tt
ParametricClass} has functions for calculating $d^2$ using both MED and GED as
well as a function for calculating the value of $p(A) \cdot P(x|A)$ as a
measure of probability for MAP classification. The {\tt NonParametricClass} contains a
function for calculating distance to the class using kNN.

\subsection{Calculations}
Distance-squared by MED is calculated in the {\tt ParametricClass} class in the
{\tt MED( point )} function. For a {\tt ParametricClass} $A$ and a point $p$,
\begin{equation}
d_{MED}^2 = (p - \mu_A)^T \cdot (p - \mu_A)
\end{equation}

Distance-squared by GED is calculated in the {\tt ParametricClass} class in the
{\tt GED( point )} function. For a {\tt ParametricClass} $A$ and a point $p$,

\begin{equation}
d_{GED}^2 = (p - \mu_A)^T \cdot \Sigma_A^{-1} \cdot (p - \mu_A)
\end{equation}

The {\tt MAP( point )} function does not really calculate distance at all. The
value returned is one side of the Bayes Theorem inequality $\bar x \in A \iff p(\bar
x|A) \cdot P(A)$ where $A$ is the class calling the function. Equation 4.16 of
the course notes gives one side of the inequality as

\begin{align*}
&P(A) \cdot \frac{1}{(2
\pi)^{n/2}} \cdot \exp(-\frac{1}{2}(p - \mu_A)^T \cdot \Sigma_A^{-1} \cdot (p
- \mu_A)) \\
= &P(A) \cdot \frac{1}{(2 \pi)^{n/2}} \cdot \exp(-\frac{1}{2} \cdot d_{GED}^2)
\end{align*}

Since the $\frac{1}{(2 \pi)^{n/2}}$ portion of the equation is constant, it can
be removed from the comparison, giving the final weighted probability as

\begin{equation}
p_{weighted} = P(A) \cdot e^{-\frac{1}{2} \cdot d_{GED}^2}
\end{equation}

Finally, kNN is calculated in {\tt NonParametricClass} in the {\tt kNN( point,
k )} function. The point $p$ is used to generate an $2 \times n$ matrix
$A$ where $A_{1j}$ is the x-coordinate and $A_{2j}$ is the y-coordinate of $p$.
An $n \times n$ matrix with the distance-squared from each point in the class
$C$ is computed by the function

$$
D = (A^T-C) \cdot (A^T - C)^T
$$

The diagonal entries of $D$ are converted to a vector, rooted and sorted. The
$k$th element is then returned as the distance.

\subsection{Plotting functions}
The classes also provide helper functions for creating graphical
representations of their data. {\tt ParametricClass} has a function for
plotting a the unit standard deviation curve and {\tt NonParametricClass}
contains a function for plotting the cluster of points the comprise the class.

\section{Static methods}
\subsection{Classification}
The classes include methods for classifying points based on the various
distance and probability methods. With the exception of the MAP classifier,
their functionality is similar. The logic is defined in Algorithm
\ref{ag:class}.

\begin{algorithm}
\caption{Classify a point based on distance to the classes}
\label{ag:class}
\begin{algorithmic}
\STATE class number = 0
\STATE minimum distance = $\infty$
\FOR{ $i$ = 1 to $n_{classes}$ }
\IF{distance to class $i \leq$ minimum distance}
\STATE class number = $i$
\STATE minimum distance = distance to class $i$
\ENDIF
\ENDFOR
\end{algorithmic}
\end{algorithm}

The difference between the function for each classification method is the
distance function that is called to determine the distance from the point to
the class. In the MAP class is that the search is for the highest weighted
probability instead of the shortest distance. Otherwise the MAP classification algorithm is similar to the rest.

\subsection{Class boundaries}
Another static method included in the classes is a function to find the class
boundaries using the different distance and probability methods. The
functions classify an $n \times m$ set of points in the x-y plane to generate
the class boundaries. The logic for these functions is defined in Algorithm
\ref{ag:bounds}.

\begin{algorithm}
\caption{Populate the matrix that determines class boundaries}
\label{ag:bounds}
\begin{algorithmic}
\STATE $C$ = an $n \times m$ matrix
\FOR{$i$ = 1 to $n$}
	\FOR{$j$ = 1 to $m$}
	$C_{ij}$ = class of the point($x_i$, $y_J$)
	\ENDFOR
\ENDFOR
\end{algorithmic}
\end{algorithm}

The function returns an $n \times m$ matrix $C$ with the elements $C_{ij} =
\{ 1,2 \ldots n_{classes} \}$. The contours of this are plotted on a graph to
reveal the boundaries of the classes.

\subsection{Class testing}
Finally, the MATLAB classes provide two functions for testing; one for
determining the confusion matrix and the other for calculating the probability
of error ($P(\varepsilon)$) given a confusion matrix. The confusion matrix
calculators generate an $n \times n$ matrix with $n$ being the number of
classes in the space. Using classes $C_i$ and test data $T_i$ with $i \in
\{1,2, \ldots, n\}$, the method is defined in Algorithm \ref{ag:confusion}.

\begin{algorithm}
\caption{Calculate the confusion matrix for a given set of classes and test
data}
\label{ag:confusion}
\begin{algorithmic}
\STATE $
M_{confusion_{n,n}} =
\begin{bmatrix}
0 & \cdots & 0 \\  
\vdots & \ddots & \vdots \\
0 & \cdots & 0 \\
\end{bmatrix}
$
\FOR{$i$ = 1 to $n_{test\_classes}$}
\FOR{$j$ = 1 to $n_{points_{i}}$}
\STATE class = the evaluated class of point $j$
\STATE add 1 to $M_{confusion}$ at the cell (class, $i$)
\ENDFOR
\ENDFOR
\end{algorithmic}
\end{algorithm}

Algorithm \ref{ag:confusion} returns the confusion matrix, $M_{confusion}$. The confusion
matrix is then used to calculate $P(\varepsilon)$ as defined in Algorithm
\ref{ag:p_err}.

\begin{algorithm}
\caption{Calculate the probability of error from a confusion matrix}
\label{ag:p_err}
\begin{algorithmic}
\STATE correct assignments = diag($M_{confusion}$)
\STATE incorrect assignments = $M_{confusion}$ - correct assignments
\STATE $P(\varepsilon)$ = $\frac{\sum \text{elements of incorrect
assignments}}{\sum \text{elements of } M_{confusion}}$
\end{algorithmic}
\end{algorithm}
