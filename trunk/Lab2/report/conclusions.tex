In the 1D and 2D cases, the nonparametric approach performed much better, in
general, than the parametric approach.  

In the 1D case, the parametric approach did produce an incredibly good
estimation when the assumed model closely matched the sample distribution. 
However, when the assumed model was incorrect and not well correlated to the
sample data, the resulting estimation was poor.

In the 2D case, similar results were observed.  The parametric approach worked
well for clusters that were close to guassian in shape, but performance
dropped significantly for clusters of unusual shape (ie crescent-shaped).  The
non-parametric approach is much more flexible because it does not assume a
standard PDF but rather estimates it directly from the data.  This makes it a
much more powerful and robust estimation method in general, and hence it is
preferred in most cases.  

The sequential discriminant approach attempts to classify the data by
sequentially combining linear discriminants that get some part of the class
exactly right.  The performance of this approach was very good, provided that the
classes are separable.  Under this condition, it is possible to get the
classification completely correct with very little overhead.