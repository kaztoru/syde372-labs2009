The implementation for Lab 1 is done using MATLAB's class structures to
maximize the reusability and allow for experiementation beyond the requirements
of the lab. This is discussed further in Chapter \ref{cha:results}.

\section{Classes}
MATLAB classes were created for parametric and non-parametric classifiers. The
classes, {\tt ParametricClass} and {\tt NonParametricClass} are presented in
Appendix \ref{cha:code}.
\subsection{Properties and class functions}
The two classes store properties related to the pattern recognition problems
they represent. The {\tt ParametricClass} stores for a class $A$ the values of
$\mu_A$, $\Sigma_A$ and $p(A)$. The {\tt NonParametricClass} stores a cluster
of $n$ points in a Gaussian distribution with the parameters $\mu$
and $\Sigma$.

Each class provides methods for calculating the various distance measures
associated with the type of problem that it represents. The {\tt
ParametricClass} has functions for calculating $d^2$ using both MED and GED as
well as a function for calculating the value of $p(A) \cdot p(x|A)$ as a
measure of probability for MAP classification. The {\tt NonParametricClass} contains a
function for calculating distance to the class using kNN.

The classes also provide helper functions for creating graphical
representations of their data. {\tt
ParametricClass}

\subsection{Static methods}

\section{Helper functions}
\subsection{Tools}
\subsection{Plot ellipse}

\section{Main file}